% +---------------------------------------------------------------+
% | Author :    Noémie Plancherel, HEIG-VD
% | Date :      18.04.2022
% +---------------------------------------------------------------+



\chapter{Étude de marché}
\label{ch:etude_marche}
Ce chapitre vise à étudier les différentes fonctionnalités offertes par les gestionnaires de mots de passe en les comparant entre plusieurs produits sélectionnés et en établissant un tableau afin d'avoir une meilleure vue d'ensemble.

Nous allons également analyser les différents prix des applications ainsi que présenter où en est le marché actuel afin d'étudier la popularité de ces dernières.

Pour l'étude de marché, les gestionnaires de mots de passe sélectionnés seront: \textit{LastPass}\footnote{\href{https://www.lastpass.com/}{https://www.lastpass.com/}}, \textit{Dashlane}\footnote{\href{https://www.dashlane.com/}{https://www.dashlane.com/}}, \textit{1Password}\footnote{\href{https://1password.com/}{https://1password.com/}}, \textit{KeePass}\footnote{\href{https://keepass.info/}{https://keepass.info/}}, \textit{Bitwarden}\footnote{\href{https://bitwarden.com/}{https://bitwarden.com/}}, \textit{NordPass}\footnote{\href{https://nordpass.com/}{https://nordpass.com/}}, \textit{Padloc}\footnote{\href{https://padloc.app/}{https://padloc.app/}}, \textit{Keeper}\footnote{\href{https://www.keepersecurity.com/}{https://www.keepersecurity.com/}}, \textit{Firefox}\footnote{\href{https://www.mozilla.org/fr/firefox/features/password-manager/}{https://www.mozilla.org/fr/firefox/features/password-manager/}}.

Ils ont été sélectionnés en se basant sur leur popularité sur le marché ainsi qu'à la suite de lecture d'articles concernant les meilleurs gestionnaires de mots de passe \cite{BPM22}\cite{gallagher19}\cite{MSPM22}\cite{PM22}. Les applications open-source ont été avantagées lors de leurs sélections.
\section{Fonctionnalités}
Ci-après, une liste des fonctionnalités disponibles dans les gestionnaires de mots de passe. Cette énumération se base sur toutes les fonctionnalités citées sur les websites des différents \textit{password manager}. \\
\begin{enumerate}
\item Stockage d'informations personnelles (cartes de crédit, passeport, contrats, etc.)
\item Remplissage automatique des formulaires en ligne (auto-complétion)
\item Partage de données entre plusieurs utilisateurs (par exemple, partage d'informations d'identifications entre une famille)
\item Générateur de mots de passe forts
\item Surveillance de la fuite de données ou données compromises
\item Alerte en cas de données compromises
\item Synchronisation de données entre devices (cloud)
\item Authentification à double facteurs
\item Self-hosting
\item Support prioritaire
\item Connexion à l'aide de facteurs biométriques (\textit{fingerprint} ou \textit{facial recognition}) ou d'un pin
\item \textit{Backup \& Restore}, possibilité de récupérer une ancienne sauvegarde

\end{enumerate}
Ci-dessous un tableau récapitulatif qui indique quels gestionnaires de mots de passe offre quelles fonctionnalités. 
\begin{longtable}[h]{|c|c|c|c|c|c|c|c|c|c|c|c|c|}
	\hline
	Application & 1 & 2 & 3 & 4 & 5 & 6 & 7 & 8 & 9 & 10 & 11 & 12 \\
	\hline
	LastPass & $\times$ & $\times$ & $\times$! & $\times$* & $\times$* &  $\times$*& $\times$ & $\times$ & & $\times$* & $\times$* &  \\
		\hline
	Dashlane & $\times$* & $\times$ & $\times$! & $\times$ & $\times$* & $\times$ & $\times$* & $\times$! & & &  & $\times$ \\
		\hline
	1Password\footnote{L'application est totalement payante et différents abonnements sont proposés \label{1p}} & $\times$* & $\times$* & $\times$* & $\times$* & & $\times$* & $\times$* & $\times$* & & $\times$* & & $\times$ \\
			\hline
	KeePass \footnote{En général, nécessite l'installation de plugins supplémentaires afin de profiter de toutes les fonctionnalités} & $\times$ & $\times$ &  & $\times$ & $\times$ & & & $\times$ & $\times$ & & $\times$ & $\times$ \\
		\hline
	Bitwarden &  & $\times$  & $\times$! & $\times$ & $\times$* & $\times$* & $\times$ & $\times$! & $\times$* &  $\times$* & $\times$! & $\times$\\
	\hline
	NordPass & $\times$ & $\times$ & $\times$*  & $\times$ & $\times$* & & $\times$ & $\times$ & $\times$* & $\times$* & $\times$ & \\
	\hline
	Padloc & $\times$ & $\times$ & $\times$* & $\times$ & $\times$* & & $\times$ & $\times$* & $\times$ & $\times$* & $\times$ & \\
	\hline
	Keeper\footnote{Se référer à la note de bas de page \ref{1p}} & $\times$* & $\times$*\footnote{Extension \textit{KeeperFill}} & $\times$* & $\times$* &$\times$* & $\times$* & $\times$* &$\times$* &$\times$* & $\times$* & $\times$* & $\times$* \\
	\hline
	Firefox &  & $\times$  &  & $\times$  & $\times$ & $\times$ & $\times$ & $\times$  &  &  &  & \\
	\hline
	\caption{Fonctionnalités proposées par les candidats}
\end{longtable} 
$\times$ : L'application propose cette fonctionnalité \\
\textbf{*}\hspace{0.1cm} : Fonctionnalité proposée mais avec un version premium (payante) \\
\textbf{!}\hspace{0.18cm} : Limitations avec une version gratuite \\

Sur tous les candidats sélectionnés, nous remarquons que la plupart offre la majorité des fonctionnalités énumérées plus haut. Nous constatons que l'offre des constructeurs de gestionnaires de mots de passe est assez variée et répond à la demande des particuliers et des entreprises.

Par rapport aux restaurations de sauvegarde, les gestionnaires cloud-based vont automatiquement créer des backups toutes les nuits, donc la restauration se fait directement dans le gestionnaire. Pour bitwarden, lorsque le gestionnaire est hébergé on-premise, il est nécessaire de créer ses propres procédures de sauvegardes. Etant donné que KeePass est uniquement en local, les sauvegardes doivent être faites manuellement et peuvent être importées sur l'application. Toutes ces solutions nécessitent le master password. Si ce dernier est oublié, il existe plusieurs solutions différentes en fonction des constructeurs (fonctionnalité pas disponible sur KeePass). 

La "sauvegarde" qui ne nécessite pas d'avoir son master password est l'exportation des données en fichier CSV, mais à moins de chiffrer le fichier, les données sont en claires ce qui n'est évidemment pas sécurisé et pas très recommandé.
\section{Plateformes}
Cette partie va permettre de visualiser sur quelles plateformes les gestionnaires de mots de passe sélectionnés sont supportés. \\
\begin{longtable}[h]{|c|c|c|c|c|c|c|}
	\hline
	Application & Windows & MacOS & Linux & Android & iOS & Navigateur  \\
	\hline
	LastPass & $\times$ & $\times$ & $\times$ & $\times$ & $\times$ &  $\times$\\
		\hline
	Dashlane & &  &  &&  & $\times$  \\
		\hline
	1Password & $\times$ & $\times$ & $\times$ & $\times$ & $\times$& \\
	\hline
	KeePass & $\times$ & $\times$* & $\times$*  & $\times$* & $\times$* &  $\times$*  \\
		\hline
	Bitwarden & $\times$ & $\times$  & $\times$ & $\times$ & $\times$ & $\times$  \\
		\hline
	NordPass & $\times$ & $\times$ & $\times$  & $\times$ & $\times$ &  \\
		\hline
	Padloc & $\times$ & $\times$ & $\times$ & $\times$ & $\times$ & $\times$ \\
		\hline
	Keeper & $\times$ & $\times$ & $\times$ & $\times$ &$\times$ & $\times$ \\
		\hline
	Firefox & &  &  &&  & $\times$  \\
		\hline
	\caption{Plateformes supportées par les différentes applications}
\end{longtable}
$\times$ : L'application est supportée sur ces plateformes \\
\textbf{*}\hspace{0.1cm} :  Des applications (ou des paquets) compatibles avec KeePass Password Safe non-officielles mais contribuées existent \\

Même si un gestionnaire supporte toutes les plateformes indiquées, il est nécessaire d'aller vérifier les conditions d'utilisation du système, c'est-à-dire les versions des plateformes afin de s'assurer que l'application fonctionnera quand même. 

Cependant, nous constatons que la majorité des applications sont disponibles sur les plateformes les plus courantes, et même si elles ne le sont pas, il y a souvent une solution non-officielle (notamment pour KeePass) ou via le navigateur qui existe.
\section{Prix}
Nous allons passer brièvement en revue les prix proposés par les gestionnaires de mots de passe. Chaque application propose leurs propres gammes de prix avec également des abonnements possibles pour les particuliers, familles ou entreprises. 
\subsection{Particuliers}
Pour la plupart des applications, nous pouvons retrouver 3 gammes de prix; Gratuit, Premium, Famille. L'offre familiale va être plus chère car les gestionnaires de mots de passe sont conçus pour pouvoir avoir plusieurs gestionnaires chiffrés individuels. 

Les tarifs ci-dessous sont exprimés en mensualités et en USD. \\
\begin{longtable}[h]{|c|c|c|c|}
		\hline
	Application & Gratuit & Premium & Famille \\
		\hline
	LastPass & \$0 & \$3 & \$4  \\
		\hline
	Dashlane & \$0 & \$3.99 & \$5.99 \\
		\hline
	1Password & non & \$2.99 & \$4.99  \\
		\hline
	KeePass\footnote{Gratuit et open-source \label{kp}} & \$0 & non & non   \\
		\hline
	Bitwarden & \$0 & <\$1 & \$3.33   \\
		\hline
	NordPass & \$0 & \$1.84 & \$4.99  \\
	\hline
	Padloc & \$0 & \$3.49 & \$5.95    \\
	\hline
	Keeper & non & \$2.92 & \$6.25  \\
	\hline
	Firefox & non & non & non  \\
	\hline
\caption{Tarifs pour particuliers}
\end{longtable}
\subsection{Entreprises}
Les entreprises ont quant à elle des prix différents dû à leurs besoins spécifiques où ils pourraient avoir besoin d'un devis personnel afin de choisir l'abonnement qui convient au mieux à leur infrastructure. Il existe plusieurs catégories qui sont en fonction du nombre d'employés et également par rapport aux fonctionnalités souhaitées. 

Chaque prix est indiqué en mensualités, en USD et par employé.
\begin{longtable}[h]{|c|c|c|}
	\hline
	Application & Team & Business \\
	\hline
	LastPass & \$4 & \$6  \\
	\hline
	Dashlane & \$5 & \$8 \\
	\hline
	1Password &  & \$7.99  \\
	\hline
	KeePass\footnote{Voir \ref{kp}} & non & non \\
	\hline
	Bitwarden & \$3 & \$5  \\
	\hline
	NordPass &  & \$3.50* \\
	\hline
	Padloc & \$3.49 & \$6.99* \\
	\hline
	Keeper & & \$3.75* \\
	\hline
	Firefox & non & non \\
    \hline
	\caption{Tarifs pour les entreprises}
\end{longtable}
\textbf{*}\hspace{0.1cm} :  Il y a la possibilité d'établir un devis en fonction des besoins spécifiques de l'entreprise \\

\section{Marché actuel}
Afin d'effectuer une étude un peu plus approfondie et afin d'établir un constat de la demande actuelle sur le marché et de leur popularité, nous allons analyser les différentes statistiques des gestionnaires de mots de passe. 

Malgré les multiples fonctionnalités proposées par les \textit{password managers}, les particuliers restent plutôt réticents à l'idée d'en utiliser un régulièrement; d'après un sondage lancé par PasswordManager\cite{PMC} aux Etats-Unis avec des personnes âgées de 18-55+, seulement 22.5\% utilisent des gestionnaires de mots de passe. Une autre étude de Security.org\cite{PM21} de novembre 2021, également lancée aux Etats-Unis, ressort les mêmes statistiques; 20\% des utilisateurs utilisent ces derniers. Les autres solutions communes pour stocker ses identifiants sont la mémorisation, le papier, la réutilisation, etc. Nous pouvons sans aucun doute déclarer que ces méthodes ne sont pas très sécurisées. 

Néanmoins, nous pouvons expliquer cette réticence à l'aide des études citées ci-dessus qui déclarent qu'au niveau des utilisateurs qui n'utilisent pas de gestionnaires de mots de passe, 70\% ne font pas confiance à la sécurité qu'elles fournissent, ils pensent que leur application pourrait être hackée. Certains, ne font également pas confiance aux constructeurs de ces dernières en pensant qu'ils volent leurs données.

En contradiction à ces avis populaires, en se basant sur un sondage de 2022 de bitwarden\cite{bitwardenreport}, globalement, 35\% des utilisateurs sont plus inquiets des cyberattaques par rapport à l'année 2020. Nous pouvons justifier ces inquiétudes avec le fait que le nombre de cyberattaques effectuées en 2021 a augmenté (en partie dû au COVID-19 et au \textit{home office}). Le DBIR de 2022 (Data Breach Investigations Report)\cite{dbir} indique qu'il y a eu une augmentation de 13\% des vols de données (dont 85\% font partie de vulnérabilités humaines).

Avec toutes ces statistiques, nous pouvons constater que malgré une utilisation encore trop basse des gestionnaires de mots de passe, les particuliers s'y intéressent progressivement dû aux attaques et vols de données en progression constante. Cependant, il y a un manque de confiance général sur ces derniers, surtout envers les constructeurs. C'est pourquoi, la sécurité parfaite au sein des gestionnaires est un sujet très important si l'on souhaite augmenter la protection des données et éviter des vulnérabilités humaines (notamment l'utilisation de mots de passe trop de simple, comme "123456"). La sécurité "presque" parfaite des applications pourraient également baisser les vols de données par des personnes malveillantes.
\section{Récapitualtif de l'étude}
Nous allons résumer toutes les informations que nous avons recueillies dans ce chapitre-ci; au final, nous constatons que les gestionnaires de mots de passe qui sont actuellement sur le marché (ici les plus populaires), sont assez complets au niveau des fonctionnalités proposées et ils sont adaptées pour tout type d'utilisation (personnelle, familiale ou professionnelle). Pour les gestionnaires payants, leurs prix sont assez abordables pour l'offre qu'ils proposent. Toutefois, même les gestionnaires en version gratuite, convient tout à fait à une utilisation quotidienne.

Au niveau des applications comparées, toutes ont leurs points positifs et leurs points négatifs (l'aspect sécuritaire et les failles connues seront discutées dans le chapitre \hyperref[ch:etude_secu]{\textit{étude sécuritaire}}). 

LastPass propose une version gratuite avec les fonctionnalités classiques que l'on attend d'un gestionnaire de mot de passe. La limite est que l'application n'est accessible que depuis un seul type d'appareil, ils font la différence entre ordinateur (fixe et portable) et appareil mobile (téléphone, montre, tablette). La version payante offre le MFA ainsi qu'un dashboard (avec les alertes de sécurité et la surveillance sur les données compromises), ce qui est une fonctionnalité intéressante.

La version gratuite de DashLane propose un stockage jusqu'à 50 mots de passe, ce qui est au final assez limtité mais il offre le 2FA ainsi que le partage sécurisé (jusqu'à 5 comptes). La version premium, permet l'utilisation d'un VPN ainsi qu'une synchronisation sur plusieurs appareils.

1Password est totalement payant mais est l'un des gestionnaires de mots de passe le plus populaire sur le marché.

KeePass est une application gratuite et open-source. Il propose une grande sélection de plugins assez utiles et variés, ce qui permet une grande offre, en plus d'être complètement gratuite.

bitwarden propose une version gratuite étonnement très complète avec un stockage illimité de mots de passe et un nombre illimité d'appareils. La version premium offre un 2FA avancé (notamment la connexion à l'aide d'une Yubikey) ainsi que des rapports de sécurité.

NordPass propose également une version gratuite complète qui permet le MFA ou encore la synchronisation entre plusieurs appareils, ce qui est très utile. Le premium propose l'aspect sécuritaire en plus. Cependant, c'est la solution gratuite la meilleure de tous les candidats sélectionnés.

Padloc n'est pas un gestionnaire très populaire mais il l'avantage d'être open-source et d'être disponible sur Github\footnote{\href{https://github.com/padloc/padloc}{https://github.com/padloc/padloc}}. La version gratuite n'offre pas beaucoup de fonctionnalités mais il y a la possibilité de stocker un nombre illimité de secrets et d'y connecter un nombre illimité d'appareils. De plus, il est multi-plateformes.

Keeper est complètement payant mais a une offre très complète et est particulièrement bien adapté pour les entreprises.

Finalement, le gestionnaire de mots de passe proposé par Firefox est directement inclus avec le navigateur, ainsi ses fonctionnalités proposées sont assez basiques et pas très poussées, mais il propose les fonctionnalités attendues d'un gestionnaire, c'est-à-dire enregistrement, génération et synchronisation de mots de passe.
