% +---------------------------------------------------------------+
% | Author :    Noémie Plancherel, HEIG-VD
% | Date :      18.04.2022
% +---------------------------------------------------------------+


\chapter{Introduction}
\label{ch:intro}

Pour un utilisateur lambda, il peut être difficile de se souvenir de tous ses mots de passe tout en s'assurant d'en utiliser un différent pour chaque service afin d'éviter tout vol de données dû à la compromission d'un service, et donc la compromission d'un seul mot de passe au lieu de plusieurs. Typiquement dans ces situations, l'utilisateur va naturellement utiliser des mots de passe simples, qui sont facilement mémorisables. Comme, par exemple, utiliser son prénom et sa date de naissance, "123456" ou encore "qwerty". De plus, il est plus simple d'utiliser le même mot de passe pour chacun de ses comptes, afin d'en mémoriser uniquement un seul. 

Cependant, même si l'unique mot de passe qu'on utilise est fort et aléatoire, il n'est pas garanti à 100\% qu'on soit la cible d'aucun attaquant et si une attaque est réalisée, toutes nos données personnelles sont exposées. Par exemple, une personne malveillante pourrait attaquer un service, comme Google, et récupérer quelques données utilisateurs, dont le mot de passe, ce qui lui permettrait par la suite d'utiliser cet unique mot de passe pour usurper l'identité de l'utilisateur sur d'autres services.

Ainsi, dans ce genre de cas, les gestionnaires de mots de passe interviennent et peuvent faciliter le quotidien de la plupart des utilisateurs. 

Néanmoins, avant d'aller plus en détails sur le fonctionnement des gestionnaires de mots de passe, nous pouvons mettre en avant la réticence des utilisateurs quant à l'utilisation régulière de ces derniers. D'après un sondage lancé par PasswordManager\cite{PMC} aux États-Unis avec des personnes âgées de 18-55+, seulement 22.5\% utilisent des gestionnaires de mots de passe. Une autre étude de Security.org\cite{PM21} de novembre 2021, également lancée aux États-Unis, ressort les mêmes statistiques; 20\% des utilisateurs utilisent ces derniers. Les autres solutions communes pour stocker ses identifiants sont, comme nous l'avons cité plus haut, la mémorisation, le papier, la réutilisation, etc. Nous pouvons sans aucun doute déclarer que ces méthodes ne sont pas très sécurisées.

Toutefois, nous pouvons expliquer cette réticence à l'aide des études citées ci-dessus qui déclarent qu'au niveau des utilisateurs qui n'utilisent pas de gestionnaires de mots de passe, 70\% ne font pas confiance à la sécurité qu'elles fournissent, ils pensent que leur application pourrait être hackée. Certains, ne font également pas confiance aux constructeurs de ces dernières en pensant qu'ils volent leurs données.

En contradiction à ces avis populaires, en se basant sur un sondage de 2022 de bitwarden\cite{bitwardenreport}, globalement, 35\% des utilisateurs sont plus inquiets des cyberattaques par rapport à l'année 2020. Nous pouvons justifier ces inquiétudes avec le fait que le nombre de cyberattaques effectuées en 2021 a augmenté (en partie dû au COVID-19 et au \textit{home office}). Le DBIR de 2022 (Data Breach Investigations Report)\cite{dbir} indique qu'il y a eu une augmentation de 13\% des vols de données (dont 85\% font partie de vulnérabilités humaines).

Avec toutes ces statistiques, nous pouvons constater que malgré une utilisation encore trop basse des gestionnaires de mots de passe, les particuliers s'y intéressent progressivement dû aux attaques et vols de données en progression constante. Cependant, il y a un manque de confiance général sur ces derniers, surtout envers les constructeurs. C'est pourquoi, la sécurité parfaite au sein des gestionnaires est un sujet très important si l'on souhaite augmenter la protection des données et éviter des vulnérabilités humaines (notamment l'utilisation de mots de passe trop de simple, comme "123456"). La sécurité "presque" parfaite des applications pourraient également baisser les vols de données par des personnes malveillantes. 

Ainsi, le travail présent s'appuie sur les statistiques ci-dessus et va essayer d'évaluer au mieux la sécurité des gestionnaires de mots de passe actuellement sur le marché afin de faire gagner la confiance des utilisateurs vis-à-vis des ces applications et de les encourager à en utiliser plus fréquemment et quotidiennement.

\section{Fonctionnement général}

Les gestionnaires de mots de passe sont des applications, parfois multi-plateformes, qui vont permettre de stocker des informations sensibles telles que des mots de passe, numéros de carte de crédit ou encore des fichiers confidentiels. On peut les comparer à des coffres forts ou un trousseau de clés avec plusieurs "clés" qui nous donnent accès à différents services.  

Ces derniers proposent un \textit{master password} et / ou une \textit{master key} qui va permettre d'accéder à l'ensemble des données secrètes. En conséquence, la sécurité repose sur un seul mot de passe principal, ce qui est très bénéfique pour les utilisateurs car ils n'ont qu'un mot de passe à retenir. Une fois l'accès à l'application, l'utilisateur a la possibilité de stocker des données, générer des mots de passe forts ainsi que se connecter à des services en ligne (remplissage de formulaire d'identification automatique).

\section{Types et fonctionnalités variées}

Différentes solutions de gestionnaires de mots de passe sont disponibles. Chacune offre des fonctionnalités variées et en plusieurs types différents en fonction du besoin de l'utilisateur et des fonctionnalités proposées. 

\subsection{Stockage des données}

Pour le stockage de toutes les données sensibles ainsi que leur synchronisation, il existe 3 solutions différentes.

\textbf{Local only} 

Il stocke toutes les données chiffrées en local sur le device de l'utilisateur. Tout est stocké sur un seul device de l'utilisateur. Ainsi, la synchronisation entre plusieurs appareils n'est pas proposée. Il est possible d'utiliser la même base de données sur d'autres devices, mais il est nécessaire d'effectuer la manipulation manuellement. 

Ces gestionnaires fonctionnent donc en mode offline (hors-ligne). Ainsi, la sécurité est plutôt bonne comparé à la solution cloud car c'est du hors-ligne, cependant si on récupère / vole le device, la sécurité pourrait être compromise car il y aurait la possibilité d'avoir accès aux informations sensibles du gestionnaire de mots de passe, via notamment une gestion de mémoire mal gérée. 

\textbf{Local / sync cloud} 

Il propose de stocker les informations en local et également de pouvoir activer la synchronisation dans le cloud afin de stocker les données sur les serveurs du constructeur. Les données sont alors stockées en local et sur le cloud. 

Sur ces gestionnaires, il est ainsi possible d'être en offline et dès que l'application est en ligne, toutes les données modifiées et / ou ajoutées sont synchronisées avec les serveurs. L'utilisateur a ainsi la possibilité d'avoir accès à ses données sur n'importe quel device (ordinateur, mobile, montre) et à tout moment.

À propos de la sécurité, elle repose sur le provider de l'application car toutes les informations sont stockées sur leurs propres serveurs et également sur les appareils de l'utilisateur.

\textbf{Cloud only} 

Il propose de stocker uniquement les données dans le cloud. Ainsi, il est nécessaire d'avoir une connexion en temps tout et le hors ligne n'est pas possible. La sécurité repose entièrement sur les serveurs du constructeur. 

\subsection{Applications}

Au niveau des applications, il existe plusieurs type d'applications différentes. Nous pouvons retrouver les suivantes.

\textbf{Applications tierces}

Elles sont crées et proposées par des constructeurs indépendants et autres que les constructeurs de systèmes d'exploitations ou navigateurs. Ces dernières peuvent être présentent sur plusieurs plateformes différentes: desktop, extension de navigateur ou encore application mobile. 

En général, ces applications ont un stockage local et cloud, ce qui leur permettent de proposer la synchronisation entre tous les appareils. 

\textbf{Navigateur (browser-based)}

Les navigateurs les plus populaires, tels que Firefox, Safari ou Chrome offrent ce gestionnaire de mots de passe qui est directement inclus dans ces derniers.
Ils vont faciliter la gestion et la sauvegarde de mots de passe de comptes de sites web.

Il y a également la possibilité de synchroniser toutes les données stockées entre tous les devices qui supportent le navigateur en question (Chrome, Firefox, Safari, etc.).

Pour certains navigateurs, les informations sont stockées et chiffrées en local sur le device de l'utilisateur. Si la synchronisation est activée, les données seront également stockées dans le cloud sur les serveurs du constructeur. Un problème de sécurité importante, et la disponibilité des mots de passe sur le navigateur, si aucun master password n'est configuré et qu'on a accès à la machine, les mots de passe sont accessibles en clair sur le navigateur.

\textbf{Système d'exploitation}

Certains systèmes d'exploitation proposent des gestionnaires de mots de passe intégrés. MacOS propose "iCloud Keychain" et Windows propose "Credential Manager" qui permettent les deux d'ajouter des identifiants et de les stocker. MacOS utilise le iCloud pour la synchronisation de toutes les données et stocke également les données en local sur l'appareil.

Ces derniers sont des solutions assez limitées car ils ne sont pas disponible pour tous les appareils. Le gestionnaire de mots de passe Windows n'est seulement accessible depuis une seule machine et celui de MacOS est uniquement disponible sur les appareils Apple. 

\section{Fonctionnalités}
\label{intro}

Les gestionnaires de mots de passe proposent diverses fonctionnalités qui permettent de faciliter l'utilisation quotidienne des utilisateurs. Une des fonctionnalités les plus intéressantes, est la génération de mots de passe forts. Sur certains gestionnaires, il est possible d'ajouter ses propres critères de génération (par exemple, 16 caractères, avec des chiffres et des lettres).

Une autre fonctionnalité proposée sur les gestionnaires en extension de navigateur, sont l'auto-complétion des champs de connexion. Lors d'une première connexion à un service, l'application va en général demander à l'utilisateur pour enregistrer les identifiants dans le gestionnaire et à chaque prochaine connexion, proposera automatiquement de remplir les champs.

Ce qui est plutôt utile avec un gestionnaire de mots de passe, c'est que certains proposent de stocker d'autres éléments que des identifiants. On a la possibilité d'également stocker des informations sensibles comme notre passeport, nos cartes de crédits ou des contrats importants. 

Une fonctionnalité existante, utile pour les familles et les entreprises, permet de partager des données entre utilisateurs. Il est possible de soit créer un identifiant à partager ou de créer un dossier partagé. 

Finalement, nous l'avons déjà cité précédemment, certaines applications permet la synchronisation entre plusieurs appareils d'un seul utilisateur. Les données sont stockées et transitent via le cloud. Cela peut grandement faciliter l'utilisation des gestionnaires car il est possible d'avoir accès à nos données autant bien sur notre mobile que sur notre PC.

\section{Sécurité}

Étant donné la réticence actuelle des utilisateurs face aux gestionnaires, il est légitime de se poser la question si ces derniers ont une bonne et forte implémentation de la sécurité, ainsi c'est la question que nous tenteront de répondre durant ce rapport. Toutefois, nous pouvons présenter quelques solutions de sécurité entreprises par les constructeurs. 

Au niveau du chiffrement, même si le gestionnaire utilise le cloud pour la synchronisation des données, tout est chiffré en local sur le device de l'utilisateur. Ainsi, toutes les données transmises dans le cloud sont chiffrées et protégées. Cela permet de garder des données chiffrées sur les serveurs des constructeurs, afin que ces derniers n'aient \textbf{aucune connaissance} des données claires. 

Étant donné que le master password n'est jamais envoyé sur les serveurs des constructeurs afin qu'ils n'aient aucune connaissance des données, certains gestionnaires permettre de récupérer leur coffre-fort si l'utilisateur perd son master password. Ils créent une clé de récupération (key recovery) qui leur est transmise dès le moment où ils activent l'option de récupération. 

Des fonctionnalités concernant la sécurité des mots de passe sont également proposées; avec un abonnement payant, il est parfois possible d'avoir une surveillance active de la fuite ou de la compromission de nos propres données sur des services. Certaines applications activent des alertes qui préviennent l'utilisateur en cas de compromission. 
