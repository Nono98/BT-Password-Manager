% +---------------------------------------------------------------+
% | Author :    Sylvain Pasini, HEIG-VD
% | Date :       June 3rd, 2021
% +---------------------------------------------------------------+


\chapter{Cahier des charges}



\section*{Résumé du problème}
De nos jours, les gestionnaires de mots de passe sont des outils très fréquemment utilisés et par la plupart des utilisateurs. Ils permettent principalement de faciliter le stockage des mots de passe qui demandent d'être de plus en plus long et complexes. Ils permettent également d'ajouter une couche sécuritaire aux mots de passe en les stockant de manière sécurisée et en offrant la possibilité de générer des mots de passe forts.

Étant donné que les utilisateurs se reposent grandement sur les gestionnaires de mot de passe, il est important de s'assurer qu'ils satisfassent un certain nombre de principes de sécurité ainsi qu'une implémentation robuste afin d'éviter tout vol ou perte de données. 



\section*{Objectifs}
Ainsi, ce travail de bachelor est séparé en deux parties distinctes; une première partie qui est une étude approfondie et complète sur les gestionnaires de mots de passe en général. Elle permet d'analyser les menaces des différents type de gestionnaires de mots de passe et de présenter les exigences sécuritaires qu'il serait nécessaire de garantir. Elle va également se concentrer sur une étude de marché avec une comparaison de plusieurs gestionnaires de mots de passe existants sous différents aspects et de sélectionner 4 candidats pour la partie pratique. L'objectif est de sélectionner des types de gestionnaires de mots de passe suivants :
\begin{enumerate}
	\item cloud
	\item local - application desktop
	\item extension de navigateur
	\item closed-source
\end{enumerate}

La deuxième partie du travail se concentre sur 4 gestionnaires de mots de passe sélectionnés dans la première partie. Chaque élément choisi est analysé et évalué en fonction de différents critères comme les choix cryptographiques utilisés, le stockage, ou encore l'architecture de l'application. Le but est d'évaluer la sécurité de manière complète de chaque gestionnaire de mot de passe sélectionné. 

Après l'analyse sécuritaire de chaque application, le rapport contient une synthèse des résultats détaillée ainsi qu'une comparaison de chaque évaluation entre-elles. 

%\subsection*{Déroulement}


\subsection*{Livrables}
Les délivrables seront les suivants :
\begin{enumerate}
\item Une documentation contenant :
	\begin{itemize}
	\item Une analyse de menaces de différents type de gestionnaire de mots de passe
	\item Spécification des exigences sécuritaires à garantir
	\item Présentation des différents type de gestionnaire de mots de passe
	\item Étude de marché
	\item Sélection des 4 candidats pour la seconde partie du travail
	\end{itemize}
\item Analyse sécuritaire des 4 candidats choisis, chaque analyse se décomposera ainsi :
	\begin{itemize}
		\item Sélection de critères d'analyse
		\item Analyse complète de chaque aspect
		\item Report des faiblesses trouvées au fabricant
	\end{itemize}
\item Synthèse des résultats
\item Comparaison entre chaque candidat
\end{enumerate}

