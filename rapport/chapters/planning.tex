% +---------------------------------------------------------------+
% | Author :    Noémie Plancherel, HEIG-VD
% | Date :      April 1st, 2022
% +---------------------------------------------------------------+


\chapter{Planning}
Le travail de bachelor sera séparé en plusieurs tâches et sous-tâches différentes qui permettront de répartir plus facilement le travail sur des périodes de plusieurs semaines. Ci-dessous, le planning détaillé avec toutes les tâches : 

\begin{enumerate}
	\item Préparation
	\begin{itemize}
		\item Rédaction du cahier des charges
		\item Planification
		\item Recherches initiales et introduction
	\end{itemize}
	\item Étude de marché
		\begin{itemize}
			\item Recherche et explication des différents types de gestionnaires de mots de passe
			\item Comparaison des fonctionnalités, du prix et des plateformes disponibles
			\item Analyse du marché actuel et de la demande
			\item Récapitulatif 
		\end{itemize}
	\item Étude sécuritaire
		\begin{itemize}
			\item Présentation de la sécurité implémentée dans les gestionnaires de mots de passe
			\item Identification et analyse des menaces potentielles
			\item Rédaction des exigences sécuritaires
		\end{itemize}
	\item Sélection
		\begin{itemize}
			\item Mise en place des critères de sélection des candidats
			\item Sélection des candidats
		\end{itemize}
	\item Analyse sécuritaire (pour chaque candidat)
		\begin{itemize}
			\item Identification et rédaction des critères d'analyse 
			\item Analyse sécuritaire de chaque aspect
		\end{itemize}
	\item Synthèse (pour chaque candidat)
		\begin{itemize}
			\item Synthèse des résultats
			\item Rapport des faiblesses au fabricant
		\end{itemize}
	\item Synthèse générale
		\begin{itemize}
			\item Comparaison de tous les résultats
			\item Conclusion du travail
		\end{itemize}
	\item Documentation
		\begin{itemize}
			\item Rédaction du rapport
			\item Lecture / visualisation de documents
			\item Tenue d'un journal de travail
		\end{itemize}
\end{enumerate}

Un diagramme de Gantt a également été effectué afin de pouvoir visualiser le planning et ajouter des périodes de temps :
\begin{figure}[h!]
	\includegraphics[width=15.5cm]{images/planning.png}
	\caption{Planning du travail de Bachelor}
\end{figure}