% +---------------------------------------------------------------+
% | Author :    Noémie Plancherel, HEIG-VD
% | Date :      18.04.2022
% +---------------------------------------------------------------+

\chapter{Étude sécuritaire}
\label{ch:etude_secu}

Ce chapitre est dédié à toute l'analyse sécuritaire des gestionnaires de mots de passes en général. Nous allons dans un premier temps décrire comment ces applications sont sécurisées en fonction de leur type, puis justifier l'importance d'une forte sécurité suite à l'augmentation de la demande des entreprises et des particuliers.

Dans un second temps, nous allons identifier et analyser toutes les menaces existantes et / ou potentielles des \textit{password manager} en mettant en avant les failles actuellement connues des constructeurs et les conséquences de ces dernières ou des faiblesses qui pourraient survenir à tout moment (par exemple des cyberattaques).

Finalement, nous allons rédiger toutes les exigences sécuritaires que doivent respecter les gestionnaires de mots de passe afin que ces dernières garantissent une utilisation sûre qui évite des pertes ou vol de données.

\section{Implémentation de la sécurité dans les gestionnaires de mots de passe}

Dans cette section, afin de se baser sur des gestionnaires de mots de passe déjà existants et de pouvoir comparer les différentes sécurités implémentées, nous allons reprendre les 8 candidats sélectionnés dans la partie \hyperref[ch:etude_marche]{\textit{étude de marché}}, c'est-à-dire; \textit{LastPass}, \textit{Dashlane}, \textit{1Password}, \textit{KeePass}, \textit{Bitwarden}, \textit{NordPass}, \textit{Padloc} et \textit{Keeper}. Toutes les informations citées sont basées sur les \textit{security whitepapers} des constructeurs\cite{lastpasssecurity}\cite{dashlanesecurity}\cite{1passwordsecurity}\cite{keepasssecurity}\cite{bitwardensecurity}\cite{padlocsecurity}\cite{keepersecurity}.

Les gestionnaires de mots de passe sélectionnés fonctionnent tous de la même manière, au final cette méthode est plutôt classique dans les architectures des applications. Un \textit{master password} (qui est seulement connu par l'utilisateur) est généré ou entré par l'utilisateur et va permettre le déverrouillage de l'application et le chiffrement / déchiffrement de toutes les données stockées. 

À part pour les gestionnaires en local qui gèrent la sécurité différemment, ils mettent en avant le \textit{Zero-knowledge encryption}. C'est une méthode qui va permettre un chiffrement \textit{end-to-end} et qui va sécuriser au mieux les données personnelles et sensibles des utilisateurs des serveurs du constructeur. En sachant que toutes les données sont stockées dans le cloud du provider, afin d'éviter que n'importe qui puisse y avoir accès, toutes les données sont chiffrées avant d'être envoyées au serveur. La clé de chiffrement reste sur le device de l'utilisateur.

Nous allons décrire dans les sous-sections suivantes comment la sécurité est implémentée dans les gestionnaires de mots de passe en fonction de leur type. 

- utilisation de la mémoire et stockage des secrets
\subsection{Les gestionnaires cloud-based}
last pass
\subsection{Les gestionnaires browser-based}
google chrome
\subsection{Les gestionnaires en local}
\subsection{Partage d'informations}
est-ce que c'est pertinent de parler de ça à ce moment ?
\subsection{Perte du master password}
\subsection{3 états du gestionnaire de mot de passe}

\subsubsection{Etat \textit{Not Running}}
\subsubsection{Etat \textit{Unlocked State}}
en expliquant chaque état et pour l'état unlock expliquer comment est géré le master password, avec des schemas
explication pour extension de navigateur, local et cloud-based
\subsubsection{Etat \textit{Locked State}}

+ facteurs biométriques !

\subsection{Algorithmes cryptographiques}
- les algos utilisés pour le chiffrement et auth des données 
\subsection{L'importance d'une forte sécurité}
\colorbox{pink}{\parbox{15cm}{à voir si utile}}
\section{Analyse des menaces}
\subsection{Failles connues des constructeurs}
\colorbox{pink}{\parbox{15cm}{à voir si je devrais pas les ajouter dans le chapitre de l'analyse de chaque gestionnaire sélectionné}}
\subsection{Conséquences d'une quelconque faiblesse}
\colorbox{pink}{\parbox{15cm}{à voir si utile, mais les conséquences seront sûrement soulignées lorsque je ferai l'analyse de menaces de toute manière}}
- remember me du master password HAA
\section{Exigences sécuritaires à respecter}