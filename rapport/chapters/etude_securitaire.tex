% +---------------------------------------------------------------+
% | Author :    Noémie Plancherel, HEIG-VD
% | Date :      18.04.2022
% +---------------------------------------------------------------+


\chapter{Étude sécuritaire}
\label{ch:etude_secu}

Ce chapitre est dédié à toutes l'analyse sécuritaire des gestionnaires de mots de passes en général. Nous allons dans un premier temps décrire comment ces applications sont sécurisées, puis justifier l'importance d'une forte sécurité suite à l'augmentation de la demande des entreprises et des particuliers.

Dans un second temps, nous allons lister et analyser toutes les menaces existantes des \textit{password manager} en mettant en avant les failles actuellement connues des constructeurs et les conséquences de ces dernières ou de faiblesses qui pourraient survenir à tout moment (par exemple des cyberattaques).

\section{Sécurité}

Dans cette section, afin de se baser sur des gestionnaires de mots de passe déjà existants et de pouvoir comparer les différentes sécurités implémentées, nous allons reprendre les 8 candidats sélectionnés dans la partie \hyperref[ch:etude_marche]{\textit{étude de marché}}, c'est-à-dire; \textit{LastPass}, \textit{Dashlane}, \textit{1Password}, \textit{Keepass}, \textit{Bitwarden}, \textit{Nordpass}, \textit{RoboForm} et \textit{Keeper}.
\subsection{Fonctionnement de la sécurité}
\subsection{L'importance d'une forte sécurité}
\section{Menaces}
\subsection{Failles connues des constructeur}
\subsection{Conséquences d'une quelconque faiblesse}